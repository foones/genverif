
\begin{proposition}[Confluence]
\label{prop:confluence}
The $\lambdaCheck$-calculus is confluent.
\end{proposition}
\begin{proof}
It suffices to note that the calculus can be encoded as an orthogonal HRS.
\end{proof}

\begin{proposition}[Weak head factorization]
\label{prop:factorization}
If $\utm \tos \iunit$ then $\utm \tows \iunit$.
\end{proposition}
\begin{proof}
\TODO{Quizás se puede probar a través de un sistema de tipos o simplemente con técnicas de reescritura.}
\end{proof}

\begin{lemma}[Uniqueness of the weak head redex]
\label{lem:uniqueness_of_weak_head_redex}
If $\ctxof{\wctx_1}{\utm_1} = \ctxof{\wctx_2}{\utm_2}$
where $\utm_1$ and $\utm_2$ are redexes, then
$\wctx_1 = \wctx_2$ and $\utm_1 = \utm_2$.
In particular, this entails determinism of weak head reduction,
\ie if $\utmtwo \tow \utmtwo_1$ and $\utmtwo \tow \utmtwo_2$
then $\utmtwo_1 = \utmtwo_2$.
\end{lemma}
\begin{proof}
We proceed by induction on $\wctx_1$:
\begin{enumerate}
\item
  $\wctx_1 = \ctxhole$: then $\utm_1 = \ctxof{\wctx_2}{\utm_2}$.
  We claim that $\wctx_2 = \ctxhole$.
  Indeed, if $\wctx_2 \neq \ctxhole$
  then $\ctxof{\wctx_2}{\utm_2}$ does not contain a redex at the root,
  as can be seen by case analysis on the shape of the weak head
  context $\wctx_2$ and the redex $\utm_2$.
  Hence $\wctx_1 = \wctx_2 = \ctxhole$ and $\utm_1 = \utm_2$.
\item
  $\wctx_1 = \wctx'_1\,\utmtwo$: note that $\wctx_2$ cannot be empty,
  because this would imply that $\wctx_1$ is empty,
  using a symmetrical argument as in the base case.
  Hence $\wctx_1$ and $\wctx_2$ are both non-empty.
  Since $\ctxof{\wctx_1}{\utm_1} = \ctxof{\wctx_2}{\utm_2}$ by hypothesis,
  then $\wctx_2$ must be of the form $\wctx_2 = \wctx'_2\,\utmtwo$
  and $\ctxof{\wctx'_1}{\utm_1} = \ctxof{\wctx'_2}{\utm_2}$.
  By \ih, $\wctx'_1 = \wctx'_2$ and $\utm_1 = \utm_2$.
  To conclude, note that
  $\wctx_1 = \wctx'_1\,\utmtwo = \wctx'_2\,\utmtwo = \wctx_2$.
\item
  $\wctx_1 = \eunit{\wctx'_1}{\utmtwo}$:
  similar to the previous case.
\item
  $\wctx_1 = \verifbtyp{\wctx'_1}$:
  similar to the previous case.
\end{enumerate}
\end{proof}

\begin{lemma}[Splitting]
\label{lem:splitting}
The following are equivalent:
\begin{enumerate}
\item
  $\ctxof{\wctx}{\eunit{\utm}{\utmtwo}} \tow^n \iunit$
\item
  There exist $n_1,n_2 \geq 0$ such that
  $\utm \tow^{n_1} \iunit$
  and $\ctxof{\wctx}{\utmtwo} \tow^{n_2} \iunit$
  and $n = 1 + n_1 + n_2$.
\end{enumerate}
\end{lemma}
\begin{proof}
($1 \implies 2$)
  Suppose that $\ctxof{\wctx}{\eunit{\utm}{\utmtwo}} \tow^n \iunit$.
  We proceed by induction on $n$.
  The reduction cannot be empty, so $n > 0$, \ie $n = 1 + m$ for some $m \geq 0$,
  and the reduction must be of the form
  $\ctxof{\wctx}{\eunit{\utm}{\utmtwo}} \tow \utmthree \tow^m \iunit$.
  We consider two subcases, depending on whether
  the redex contracted in the first step is $\eunit{\utm}{\utmtwo}$
  or internal to $\utm$:
  \begin{enumerate}
  \item
    If the contracted redex is $\eunit{\utm}{\utmtwo}$,
    then $\utm = \iunit$,
    and the reduction is of the form
    $\ctxof{\wctx}{\eunit{\iunit}{\utmtwo}}
     \tow \ctxof{\wctx}{\utmtwo}
     \tow^m \iunit$.
    Hence $\utm = \iunit \tow^0 \iunit$
    and $\ctxof{\wctx}{\utmtwo} \tow^m \iunit$,
    and taking $n_1 := 0$ and $n_2 := m$ we have that
    $n = 1 + m = 1 + n_1 + n_2$, as required.
  \item
    If the contracted redex is internal to $\utm$,
    then the reduction is of the form
    $\ctxof{\wctx}{\eunit{\utm}{\utmtwo}}
     \tow \ctxof{\wctx}{\eunit{\utm'}{\utmtwo}}
     \tow^m \iunit$
    where $\utm \tow \utm'$.
    By \ih there exist $m_1,m_2$
    such that $\utm' \tow^{m_1} \iunit$
    and $\ctxof{\wctx}{\utmtwo} \tow^{m_2} \iunit$
    and $m = 1 + m_1 + m_2$.
    Taking $n_1 := 1 + m_1$ and $n_2 := m_2$,
    we have that
    $\utm \tow \utm' \tows^{m_1} \iunit$
    where $n = 1 + m = 2 + m_1 + m_2 = 1 + n_1 + n_2$.
  \end{enumerate}
($2 \implies 1$)
  Suppose that $\utm \tow^{n_1} \iunit$ and $\ctxof{\wctx}{\utmtwo} \tow^{n_2} \iunit$.
  Then we have that
  $\ctxof{\wctx}{\eunit{\utm}{\utmtwo}}
  \tow^{n_1} \ctxof{\wctx}{\eunit{\iunit}{\utmtwo}}
  \tow \ctxof{\wctx}{\utmtwo}
  \tow^{n_2} \iunit$
  of length $n = 1 + n_1 + n_2$.
\end{proof}

\begin{definition}[Substitution]
A \emph{substitution}, ranged over by $\subst,\substtwo,\hdots$,
is a partial function mapping variables to (untyped) terms.
If $\subst$ is a substitution, we write $\subst\extsub{\var}{\utm}$ for
the extension of the substitution with the mapping $\var\mapsto\utm$, \ie:
\[
  (\subst\extsub{\var}{\utm})(\vartwo)
  \eqdef
  \begin{cases}
    \utm            & \text{if $\vartwo = \var$} \\
    \subst(\vartwo) & \text{otherwise}
  \end{cases}
\]
If $\utm$ is an untyped term and $\subst$ is a substitution,
we write $\subs{\utm}{\subst}$ for the term obtained by
substituting each free variable in $\utm$ by the corresponding
term in $\subst$, avoiding capture. More precisely:
\[
  \begin{array}{rcll}
    \subs{\var}{\subst}
  & \eqdef &
    \begin{cases}
      \subst(\var) & \text{if $\var \in \dom{\subst}$} \\
      \var         & \text{otherwise}
    \end{cases}
  \\
    \subs{(\lam{\var}{\utm})}{\subst}
  & \eqdef &
    \lam{\var}{\subs{\utm}}{\subst}
    & \text{if $\var \notin \dom{\subst}$}
  \\
    \subs{(\utm\,\utmtwo)}{\subst}
  & \eqdef &
    \subs{\utm}{\subst}\,\subs{\utmtwo}{\subst}
  \\
    \subs{\iunit}{\subst}
  & \eqdef &
    \iunit
  \\
    \subs{(\eunit{\utm}{\utmtwo})}{\subst}
  & \eqdef &
    \eunit{\subs{\utm}{\subst}}{\subs{\utmtwo}{\subst}}
  \\
    \subs{\genbtyp}{\subst}
  & \eqdef &
    \genbtyp
  \\
    \subs{(\verifbtyp{\utm})}{\subst}
  & \eqdef &
    \verifbtyp{\subs{\utm}{\subst}}
  \end{array}
\]
\end{definition}

\begin{lemma}[Substitution commutes with redexes]
\label{lem:subst_commute_redex}
\label{lem:tow_deterministic}
If $\utmfive \to \utmsix$ is a reduction step at the root,
then $\subs{\utmfive}{\subst} \to \subs{\utmsix}{\subst}$
is a reduction step at the root. 
\end{lemma}
\begin{proof}
Straightforward, by case analysis on the possible shapes of the redex $\utmfive$.
\end{proof}

\begin{lemma}[Substitution commutes with weak head contexts]
\label{lem:subst_commute_wctx}
\quad
\begin{enumerate}
\item
  If $\wctx$ is a weak head context,
  then $\subs{\wctx}{\subst}$ is a weak head context. 
\item
  $\subs{\ctxof{\wctx}{\utm}}{\subst}
  = \ctxof{\subs{\wctx}{\subst}}{\subs{\utm}{\subst}}$
\end{enumerate}
\end{lemma}
\begin{proof}
Both items are straightforward by induction on $\wctx$.
The key remark is that the hole of a weak head context does not lie
inside an abstraction, so in the expression
$\ctxof{\subs{\wctx}{\subst}}{\subs{\utm}{\subst}}$ there is no
danger of variable capture.
\end{proof}

\begin{definition}[Generative substitution]
Let $\tenv$ be a type environment and $\utm$ an untyped term.
We write $\subs{\utm}{\tenv}$ to stand for $\subs{\utm}{\subst}$,
where $\subst$ is the substitution such that
$\subst(\var) = \gen{\typ}$ for all $\var \in \dom{\tenv}$,
and $\subst(\var)$ is undefined for all $\var \notin \dom{\tenv}$.
\end{definition}

\begin{example}
If $\tenv$ is the type environment $\var:\typ\to\typtwo,\vartwo:\typ$,
then
$\subs{(\var\,\vartwo)}{\tenv}
= \gen{\typ\to\typtwo}\,\gen{\typ}
= (\lam{\var}{\eunit{\verif{\typ}\var}{\gen{\typtwo}}})\,\gen{\typ}
$.
\end{example}

\begin{definition}[Compatibility]
Let $\tenv$ be a type environment and $\subst$ be a substitution.
We say that $\tenv$ is \defn{compatible} with $\subst$,
written $\compat{\tenv}{\subst}$,
if for every $\var:\typ \in \tenv$ we have that $\var \in \dom{\subst}$
and $\verif{\typ}{\subs{\var}{\subst}} \tos \iunit$.
\end{definition}

\begin{lemma}
\label{lem:subst_weak_head_redex}
If $\subs{\utm}{\tenv}$ is of the form $\ctxof{\wctx}{\utmfive}$,
where $\utmfive$ is a redex, then exactly one of the following holds:
\begin{enumerate}
\item
  There exist $\wctx_0,\utmfive_0$ such that
  $\utm = \ctxof{\wctx_0}{\utmfive_0}$ 
  where $\subs{\wctx_0}{\tenv} = \wctx$
  and $\subs{\utmfive_0}{\tenv} = \utmfive$
  and $\utmfive_0$ is a redex.
\item
  There exist $\wctx_0,\var,\utmtwo,\vartwo,\utmthree$ such that
  $\utm = \ctxof{\wctx_0}{\var\,\utmtwo}$
  where $\subs{\wctx_0}{\tenv} = \wctx$
  and $\subs{\var}{\tenv} = \lam{\vartwo}{\utmthree}$.
\item
  There exist $\wctx_0,\var,\utmtwo$ such that
  $\utm = \ctxof{\wctx_0}{\eunit{\var}{\utmtwo}}$
  where $\subs{\wctx_0}{\tenv} = \wctx$
  and $\subs{\var}{\tenv} = \iunit$.
\item
  There exist $\wctx_0,\var$ such that
  $\utm = \ctxof{\wctx_0}{\verifbtyp{\var}}$
  where $\subs{\wctx_0}{\tenv} = \wctx$
  and $\subs{\var}{\tenv} = \genbtyp$.
\end{enumerate}
\end{lemma}
\begin{proof}
\TODO{TODO}
\end{proof}

\begin{lemma}[Main lemma]
\label{lem:main_lemma}
Let $\compat{\tenv}{\subst}$.
If $\subs{\utm}{\tenv} \tows \iunit$
then $\subs{\utm}{\subst} \tows \iunit$.
\end{lemma}
\begin{proof}
Suppose that $\compat{\tenv}{\subst}$
and $\subs{\utm}{\tenv} \tows \iunit$.
Let us write $\typesize{\typ}$ for the size of the type $\typ$,
and $\typesize{\tenv}$ for the maximum size of a type occurring in $\tenv$,
\ie $\typesize{\tenv} \eqdef \max\set{\typesize{\typ} \ST \exists \var.\, (\var:\typ \in \tenv)}$.
We show that $\subs{\utm}{\subst} \tows \iunit$
by induction on the lexicographic measure $(\typesize{\tenv}, n)$,
where $n$ is the length of the reduction sequence $\subs{\utm}{\tenv} \tow^n \iunit$.
We consider two cases, depending on whether this reduction sequence is empty or
not, \ie $n = 0$ or $n > 0$.
\begin{enumerate}
\item
  \textbf{Empty ($n = 0$).}
  Then $\subs{\utm}{\tenv} = \iunit$.
  We claim that $\utm = \iunit$.
  Indeed, note that $\utm$ cannot be a variable, because if
  $\utm = \var$ is a variable
  then either $\var:\typ \in \tenv$
  and $\subs{\utm}{\tenv} = \gen{\typ} \neq \iunit$,
  or $\var \notin \dom{\tenv}$ and $\subs{\utm}{\tenv} = \var \neq \iunit$.
  Moreover, if $\utm$ is not $\iunit$ nor a variable, then
  $\subs{\utm}{\tenv} \neq \iunit$
  (for example, if $\utm$ is an application then
  $\subs{\utm}{\tenv}$ is also an application).
  So we have that $\utm = \iunit$ and it is immediate to conclude that
  $\subs{\utm}{\subst} = \iunit \tows \iunit$.
\item
  \textbf{Non-empty ($n = 1 + m$).}
  Since the reduction sequence $\subs{\utm}{\tenv} \tow^n \iunit$ is non-empty,
  $\subs{\utm}{\tenv}$ has a weak head redex.
  By \cref{lem:subst_weak_head_redex} there are four cases:
  either $\utm = \ctxof{\wctx}{\utmfive}$ where $\utmfive$ is a redex,
  or $\utm = \ctxof{\wctx}{\var\,\utmtwo}$ where $\subs{\var}{\tenv}$ is an abstraction,
  or $\utm = \ctxof{\wctx}{\eunit{\var}{\utmtwo}}$ where $\subs{\var}{\tenv} = \iunit$,
  or $\utm = \ctxof{\wctx}{\verifbtyp{\var}}$ where $\subs{\var}{\tenv} = \genbtyp$.
  We consider each of these four cases individually:
  \begin{enumerate}
  \item
    If $\utm = \ctxof{\wctx}{\utmfive}$ where $\utmfive$ is a redex:
    consider the root reduction step $\utmfive \tow \utmsix$.
    First note that $\subs{\utmfive}{\tenv} \tow \subs{\utmsix}{\tenv}$
    is also a root reduction step by \cref{lem:subst_commute_redex}.
    Since head reduction is deterministic (\cref{lem:tow_deterministic}),
    the input reduction sequence must be of the form:
    \[
      \begin{array}{rlll}
        \subs{\utm}{\tenv}
      & = &
        \ctxof{\subs{\wctx}{\tenv}}{\subs{\utmfive}{\tenv}}
        & \text{by \cref{lem:subst_commute_wctx}}
      \\
      & \tow &
        \ctxof{\subs{\wctx}{\tenv}}{\subs{\utmsix}{\tenv}}
        & \text{by \cref{lem:subst_commute_wctx}, using that $\subs{\utmfive}{\tenv} \tow \subs{\utmsix}{\tenv}$}
      \\
      & \tow^m &
        \iunit
      \end{array}
    \]
    The reduction sequence
    $\ctxof{\subs{\wctx}{\tenv}}{\subs{\utmsix}{\tenv}} \tow^m \iunit$
    is shorter than the original one ($m < n$).
    We may apply the \ih, using the same type environment $\tenv$,
    in such a way that the lexicographic measure decreases:
    $(\typesize{\tenv},n) > (\typesize{\tenv},m)$.
    By \ih we have that
    $\ctxof{\subs{\wctx}{\subst}}{\subs{\utmsix}{\subst}} \tows \iunit$.
    To conclude, note that
    $\subs{\utmfive}{\subst} \tow \subs{\utmsix}{\subst}$
    is a root reduction step by \cref{lem:subst_commute_redex},
    so:
    \[
      \begin{array}{rlll}
        \subs{\utm}{\subst}
      & = &
        \ctxof{\subs{\wctx}{\subst}}{\subs{\utmfive}{\subst}}
        & \text{by \cref{lem:subst_commute_wctx}}
      \\
      & \tow &
        \ctxof{\subs{\wctx}{\subst}}{\subs{\utmsix}{\subst}}
        & \text{by \cref{lem:subst_commute_wctx}, using that $\subs{\utmfive}{\subst} \tow \subs{\utmsix}{\subst}$}
      \\
      & \tows &
        \iunit
        & \text{as obtained from the \ih}
      \end{array}
    \]
  \item
    If $\utm = \ctxof{\wctx}{\var\,\utmtwo}$
    where $\subs{\var}{\tenv}$ is an abstraction,
    \ie $\subs{\var}{\tenv} = \lam{\vartwo}{\utmthree}$.

    First, we claim that $\var : \typ\to\typtwo \in \tenv$
    for certain types $\typ,\typtwo$.
    Indeed, if $\var \notin \dom{\tenv}$ then $\subs{\var}{\tenv} = \var$
    is not an abstraction; so we must have that $\var \in \dom{\tenv}$.
    Moreover if $\var : \btyp \in \tenv$, \ie if its type is an atomic type,
    then $\subs{\var}{\tenv} = \genbtyp$ is not an abstraction;
    so we must have that $\var$ occurs in $\tenv$ with a type which is not atomic.
    Thus $\var$ must occur in $\tenv$ with an arrow type.

    Then the input reduction sequence must be of the form:
    \[
      \begin{array}{rlll}
        \subs{\utm}{\tenv}
      & = &
        \subs{\ctxof{\wctx}{\var\,\utmtwo}}{\tenv}
      \\
      & = &
        \ctxof{\subs{\wctx}{\tenv}}{\subs{\var}{\tenv}\,\subs{\utmtwo}{\tenv}}
        & \text{by \cref{lem:subst_commute_wctx}}
      \\
      & = &
        \ctxof{\subs{\wctx}{\tenv}}{\gen{\typ\to\typtwo}\,\subs{\utmtwo}{\tenv}}
        & \text{since $\var:\typ\to\typtwo \in \tenv$}
      \\
      & = &
        \ctxof{\subs{\wctx}{\tenv}}{(\lam{\var}{\eunit{\verif{\typ}{\var}}{\gen{\typtwo}}})\,\subs{\utmtwo}{\tenv}}
        & \text{by definition of $\gen{\typ\to\typtwo}$}
      \\
      & \tow &
        \ctxof{\subs{\wctx}{\tenv}}{\eunit{\verif{\typ}{\subs{\utmtwo}{\tenv}}}{\gen{\typtwo}}}
        & \text{by $\symBeta$-reduction}
      \\
      & \tow^m &
        \iunit
      \end{array}
    \]
    Consider the tail of the reduction sequence:
    $\ctxof{\subs{\wctx}{\tenv}}{\eunit{\verif{\typ}{\subs{\utmtwo}{\tenv}}}{\gen{\typtwo}}}
     \tow^m \iunit$.
    By \cref{lem:splitting}, there exist $m_1,m_2 \geq 0$ such that
    $\verif{\typ}{\subs{\utmtwo}{\tenv}} \tow^{m_1} \iunit$
    and
    $\ctxof{\subs{\wctx}{\tenv}}{\gen{\typtwo}} \tow^{m_2} \iunit$
    and $m = 1 + m_1 + m_2$.

    Since $\compat{\tenv}{\subst}$ by hypothesis
    and $\var : \typ\to\typtwo \in \tenv$,
    we know that $\verif{\typ\to\typtwo}{\subs{\var}{\subst}} \tos \iunit$.
    By definition of $\verif{\typ\to\typtwo}{\ARG}$, we have:
    \[
      \verif{\typtwo}{(\subs{\var}{\subst}\,\gen{\typ})}
      = \verif{\typ\to\typtwo}{\subs{\var}{\subst}}
      \tos \iunit
    \]

    Let $\vartwo$ be a fresh variable, and consider
    the term $\ctxof{\wctx}{\vartwo}$,
    the type environment $\tenv_1 := \tenv\cup\set{\vartwo:\typtwo}$,
    and the substitution $\subst_1 := \subst\extsub{\vartwo}{\subs{\var}{\subst}\,\gen{\typ}}$.
    Note that $\compat{\tenv_1}{\subst_1}$
    because for $\vartwo:\typtwo\in\tenv_1$
    we have that
    $\verif{\typtwo}{(\vartwo^{\subst_1})}
    = \verif{\typtwo}{(\var^{\subst}\,\gen{\typ})}
    \tows \iunit$,
    and for any variable other than $\vartwo$ the property
    follows from the fact that $\compat{\tenv}{\subst}$.
    Recall that $\var:\typ\to\typtwo \in \tenv$
    so $\typesize{\tenv} \geq \typesize{\typ\to\typtwo} > \typesize{\typtwo}$,
    which means that
    $\typesize{\tenv_1}
     = \typesize{\tenv,\vartwo:\typtwo}
     = \max\set{\typesize{\tenv},\typesize{\typtwo}}
     = \typesize{\tenv}$.
    Also, note that:
    \[
      \begin{array}{rlll}
        \subs{\ctxof{\wctx}{\vartwo}}{\tenv_1}
      & = &
        \ctxof{\subs{\wctx}{\tenv_1}}{\subs{\vartwo}{\tenv_1}}
        & \text{by \cref{lem:subst_commute_wctx}}
      \\
      & = &
        \ctxof{\subs{\wctx}{\tenv}}{\subs{\vartwo}{\tenv_1}}
        & \text{since $\vartwo$ is fresh for $\wctx$}
      \\
      & = &
        \ctxof{\subs{\wctx}{\tenv}}{\gen{\typtwo}}
        & \text{by definition of $\subs{\vartwo}{\tenv_1}$}
      \\
      & \tow^{m_2} &
        \iunit
        & \text{as already shown}
      \end{array}
    \]
    Furthermore, note that the lexicographic measure decreases:
    $(\typesize{\tenv},n)
     = (\typesize{\tenv_1},n)
     > (\typesize{\tenv_1},m_2)$,
    so we may apply the \ih to obtain that
    $\subs{\ctxof{\wctx}{\vartwo}}{\subst_1} \tows \iunit$.
    Hence:
    \[
      \begin{array}{rlll}
        \ctxof{\subs{\wctx}{\subst}}{\subs{\var}{\subst}\,\gen{\typ}}
      & = &
        \ctxof{\subs{\wctx}{\subst}}{\subs{\vartwo}{\subst_1}}
        & \text{by definition of $\subs{\vartwo}{\subst_1}$}
      \\
      & = &
        \ctxof{\subs{\wctx}{\subst_1}}{\subs{\vartwo}{\subst_1}}
        & \text{since $\vartwo$ is fresh for $\wctx$}
      \\
      & = &
        \subs{\ctxof{\wctx}{\vartwo}}{\subst_1}
        & \text{by \cref{lem:subst_commute_wctx}}
      \\
      & \tows &
        \iunit
        & \text{as just shown}
      \end{array}
    \]

    Recall that we have $\verif{\typ}{\subs{\utmtwo}{\tenv}} \tow^{n_1} \iunit$.
    Taking the environment $\tenv$ and the substitution $\subst$,
    we can note that the lexicographic measure decreases:
    $(\typesize{\tenv},n) > (\typesize{\tenv_1},m_1)$,
    so may apply the \ih to obtain that
    $\verif{\typ}{\subs{\utmtwo}{\subst}} \tows \iunit$.
    Now let $\varthree$ be a fresh variable, and consider
    the term $\ctxof{\subs{\wctx}{\subst}}{\subs{\var}{\subst}\,\varthree}$,
    the type environment $\tenv_2 := \set{\varthree:\typ}$,
    and the substitution $\subst_2 := \extsub{\varthree}{\subs{\utmtwo}{\subst}}$.
    Note that $\compat{\tenv_2}{\subst_2}$
    because $\verif{\typ}{\subs{\utmtwo}{\subst}} \tows \iunit$, as we have
    just proved.
    Moreover, we have that:
    \[
      \begin{array}{rlll}
        \subs{\ctxof{\subs{\wctx}{\subst}}{\subs{\var}{\subst}\,\varthree}}{\tenv_2}
      & = &
        \ctxof{\subs{(\subs{\wctx}{\subst})}{\tenv_2}}{\subs{(\subs{\var}{\subst})}{\tenv_2}\,\subs{\varthree}{\tenv_2}}
        & \text{by \cref{lem:subst_commute_wctx}}
      \\
      & = &
        \ctxof{\subs{\wctx}{\subst}}{\subs{\var}{\subst}\,\subs{\varthree}{\tenv_2}}
        & \text{since $\varthree$ is fresh for $\wctx,\var,\subst$}
      \\
      & = &
        \ctxof{\subs{\wctx}{\subst}}{\subs{\var}{\subst}\,\gen{\typ}}
        & \text{since $\varthree : \typ \in \tenv_2$}
      \\
      & \tows &
        \iunit
        & \text{as already shown}
      \end{array}
    \]
    Let the length of this reduction sequence be called $n'$.
    Note that the lexicographic measure decreases:
    $(\typesize{\tenv},n) > (\typesize{\tenv_2},n')$
    because the first component decreases, \ie
    $\typesize{\tenv}
     \geq \typesize{\typ\to\typtwo}
     > \typesize{\typ}
     = \typesize{\tenv_2}$,
    so we may apply the \ih to conclude that
    $\subs{\ctxof{\subs{\wctx}{\subst}}{\subs{\var}{\subst}\,\varthree}}{\subst_2}
     \tows \iunit$.
    Finally, note that:
    \[
      \begin{array}{rlll}
        \subs{\utm}{\subst}
      & = &
        \subs{\ctxof{\wctx}{\var\,\utmtwo}}{\subst}
      \\
      & = &
        \ctxof{\subs{\wctx}{\subst}}{\subs{\var}{\subst}\,\subs{\utmtwo}{\subst}}
        & \text{by \cref{lem:subst_commute_wctx}}
      \\
      & = &
        \ctxof{\subs{\wctx}{\subst}}{\subs{\var}{\subst}\,\subs{\varthree}{\subst_2}}
        & \text{by definition of $\subst_2$}
      \\
      & = &
        \ctxof{\subs{(\subs{\wctx}{\subst})}{\subst_2}}{\subs{(\subs{\var}{\subst})}{\subst_2}\,\subs{\varthree}{\subst_2}}
        & \text{since $\varthree$ is fresh for $\wctx,\var,\subst$}
      \\
      & = &
        \subs{\ctxof{\subs{\wctx}{\subst}}{\subs{\var}{\subst}\,\varthree}}{\subst_2}
        & \text{by \cref{lem:subst_commute_wctx}}
      \\
      & \tows &
        \iunit
        & \text{as just shown}
      \end{array}
    \]
  \item
    If $\utm = \ctxof{\wctx}{\eunit{\var}{\utmtwo}}$
    where $\subs{\var}{\tenv} = \iunit$:
    we claim that this case is impossible. 
    Indeed, since $\subs{\var}{\tenv} = \iunit$,
    we know that $\var \in \dom{\tenv}$,
    but if $\var : \typ \in \dom{\tenv}$
    then $\subs{\var}{\tenv} = \gen{\typ} \neq \iunit$,
    contradicting the fact that $\subs{\var}{\tenv} = \iunit$.
  \item
    If $\utm = \ctxof{\wctx}{\verifbtyp{\var}}$
    where $\subs{\var}{\tenv} = \genbtyp$:
    then it must be the case that $\var : \btyp \in \tenv$.
    Since head reduction is deterministic (\cref{lem:tow_deterministic}),
    the input reduction sequence must be of the form:
    \[
      \begin{array}{rlll}
        \subs{\utm}{\tenv}
      & = &
        \subs{\ctxof{\wctx}{\verifbtyp{\var}}}{\tenv}
      \\
      & = &
        \ctxof{\subs{\wctx}{\tenv}}{\verifbtyp{\subs{\var}{\tenv}}}
        & \text{by \cref{lem:subst_commute_wctx}}
      \\
      & = &
        \ctxof{\subs{\wctx}{\tenv}}{\verifbtyp{\genbtyp}}
        & \text{since $\var : \btyp \in \tenv$}
      \\
      & \tow &
        \ctxof{\subs{\wctx}{\tenv}}{\iunit}
        & \text{by $\symVerifGen$-reduction}
      \\
      & = &
        \subs{\ctxof{\wctx}{\iunit}}{\tenv}
        & \text{by \cref{lem:subst_commute_wctx}}
      \\
      & \tow^m &
        \iunit
      \end{array}
    \]
    Note that the lexicographic measure decreases:
    $(\typesize{\tenv},n) > (\typesize{\tenv},m)$,
    so by \ih we have that
    $\subs{\ctxof{\wctx}{\iunit}}{\subst} \tows \iunit$.
    Hence:
    \[
      \begin{array}{rlll}
        \subs{\utm}{\subst}
      & = &
        \subs{\ctxof{\wctx}{\verifbtyp{\var}}}{\subst}
      \\
      & = &
        \ctxof{\subs{\wctx}{\subst}}{\verifbtyp{\subs{\var}{\subst}}}
        & \text{by \cref{lem:subst_commute_wctx}}
      \\
      & = &
        \ctxof{\subs{\wctx}{\subst}}{\iunit}
        & \text{because $\verifbtyp{\subs{\var}{\subst}} \tows \iunit$ by $\compat{\tenv}{\subst}$}
      \\
      & = &
        \subs{\ctxof{\wctx}{\iunit}}{\subst}
        & \text{by \cref{lem:subst_commute_wctx}}
      \\
      & \tows &
        \iunit
        & \text{as just shown}
      \end{array}
    \]
  \end{enumerate}
\end{enumerate}
\end{proof}

\begin{lemma}[Adequacy]
\label{lem:adequacy}
Let $\compat{\tenv}{\subst}$.
If $\subs{\utm}{\tenv} \tos \iunit$
then $\subs{\utm}{\subst} \tos \iunit$.
\end{lemma}
\begin{proof}
Suppose that $\compat{\tenv}{\subst}$ and $\subs{\utm}{\tenv} \tos \iunit$.
By \cref{prop:factorization}, we have that $\subs{\utm}{\tenv} \tows \iunit$,
so it suffices to apply \cref{lem:main_lemma}.
\end{proof}

\begin{lemma}[Generate/verify]
\label{lem:generate_verify}
$\verif{\typ}{\gen{\typ}} \tos \iunit$
\end{lemma}
\begin{proof}
By induction on $\typ$:
\begin{enumerate}
\item $\typ = \btyp$:
  Then
  $\verif{\typ}{\gen{\typ}}
  = \verifbtyp{\genbtyp}
  \to \iunit$ by rule $\symVerifGen$.
\item $\typ = \typtwo \to \typthree$:
  Then:
  \[
    \begin{array}{rlll}
      \verif{\typ}{\gen{\typ}}
    & = &
      \verif{\typtwo\to\typthree}{\gen{\typtwo\to\typthree}}
    \\
    & = &
      \verif{\typthree}{(\gen{\typtwo\to\typthree}\,\gen{\typtwo})}
    \\
    & = &
      \verif{\typthree}{((\lam{\var}{\eunit{\verif{\typtwo}{\var}}{\gen{\typthree}}})\,\gen{\typtwo})}
    \\
    & \to &
      \verif{\typthree}{(\eunit{\verif{\typtwo}{\gen{\typtwo}}}{\gen{\typthree}})}
    \\
    & \tos &
      \verif{\typthree}{(\eunit{\iunit}{\gen{\typthree}})}
      & \text{by \ih}
    \\
    & \to &
      \verif{\typthree}{\gen{\typthree}}
    \\
    & \tos &
      \iunit
      & \text{by \ih}
    \end{array}
  \]
\end{enumerate}
\end{proof}

\begin{theorem}[Soundness]
If $\judg{\tenv}{\tm}{\typ}$ is derivable,
then $\verif{\typ}{\subs{\tm}{\tenv}} \tos \iunit$.
\end{theorem}
\begin{proof}
We proceed by induction on the derivation of the judgment $\judg{\tenv}{\tm}{\typ}$.
\begin{enumerate}
\item
  Variable, $\tm = \var$:
  then $\judg{\tenv}{\var}{\typ}$ holds, where $\var:\typ\in\tenv$.
  Then
  $\verif{\typ}{\subs{\tm}{\tenv}}
  = \verif{\typ}{\subs{\var}{\tenv}}
  = \verif{\typ}{\gen{\typ}}
  \tos \iunit$
  by \cref{lem:generate_verify}.
\item
  Abstraction, $\tm = \lam{\var}{\tmtwo}$:
  then we have that $\typ = \typtwo\to\typthree$,
  and $\judg{\tenv}{\lam{\var}{\tmtwo}}{\typtwo\to\typthree}$
  is derived from $\judg{\tenv,\var:\typtwo}{\tmtwo}{\typthree}$.
  By $\alpha$-conversion, we may assume that $\var \notin \dom{\tenv}$.
  Then:
  \[
    \begin{array}{rlll}
      \verif{\typtwo\to\typthree}{\subs{(\lam{\var}{\tmtwo})}{\tenv}}
    & = &
      \verif{\typtwo\to\typthree}{\lam{\var}{\subs{\tmtwo}{\tenv}}}
    \\
    & = &
      \verif{\typthree}{((\lam{\var}{\subs{\tmtwo}{\tenv}})\,\gen{\typtwo})}
    \\
    & \to &
      \verif{\typthree}{(\subs{\tmtwo}{\tenv}\sub{\var}{\gen{\typtwo}})}
    \\
    & = &
      \verif{\typthree}{\subs{\tmtwo}{\tenv,\var:\typtwo}}
    \\
    & \tos &
      \iunit
      & \text{by \ih}
    \end{array}
  \]
\item
  Application, $\tm = \tmtwo\,\tmthree$:
  then $\judg{\tenv}{\tmtwo\,\tmthree}{\typ}$
  is derived from $\judg{\tenv}{\tmtwo}{\typtwo\to\typ}$
  and $\judg{\tenv}{\tmthree}{\typtwo}$.
  By \ih on the first premise, we have that
  $\verif{\typtwo\to\typ}{\subs{\tmtwo}{\tenv}} \tos \iunit$,
  which by definition of $\verif{\typtwo\to\typ}{\ARG}$
  means that
  $\verif{\typ}{(\subs{\tmtwo}{\tenv}\,\gen{\typtwo})} \tos \iunit$.
  By \ih on the second premise, we have that
  $\verif{\typtwo}{\subs{\tmthree}{\tenv}} \tos \iunit$.
  Let $\vartwo$ be a fresh variable,
  consider the term
  $\verif{\typ}{(\subs{\tmtwo}{\tenv}\,\vartwo)}$,
  the typing context $\tenvtwo := \set{\vartwo:\typtwo}$,
  and the substitution $\subst := \set{\vartwo \mapsto \subs{\tmthree}{\tenv}}$.
  Remark that
  $\compat{\tenvtwo}{\subst}$
  so by adequacy (\ref{lem:adequacy})
  we have that the following implication holds:
  \[
    \subs{(\verif{\typ}{(\subs{\tmtwo}{\tenv}\,\vartwo)})}{\tenvtwo} \tos \iunit
    \HS\text{ implies }\HS
    \subs{(\verif{\typ}{(\subs{\tmtwo}{\tenv}\,\vartwo)})}{\subst} \tos \iunit
  \]
  Performing the substitutions, this means that:
  \[
    \verif{\typ}{(\subs{\tmtwo}{\tenv}\,\gen{\typtwo})} \tos \iunit
    \HS\text{ implies }\HS
    \verif{\typ}{(\subs{\tmtwo}{\tenv}\,\subs{\tmthree}{\tenv})} \tos \iunit
  \]
  The antecedent has already been shown to hold,
  so
  $\verif{\typ}{\subs{\tm}{\tenv}}
   = \verif{\typ}{(\subs{\tmtwo}{\tenv}\,\subs{\tmthree}{\tenv})}
   \tos \iunit$,
  as required.
\end{enumerate}
\end{proof}

\begin{definition}[Size of a term]
Let $\size{\utm}$ denote the size of the term $\utm$.
Let and let $\sizeVar{\utm}$ be defined as the size of $\utm$
but possibly ignoring variable arguments:
\[
  \sizeVar{\utm} =
    \begin{cases}
      \sizeVar{\utm'} & \text{if $\utm = \utm'\,\var$} \\
      \size{\utm}     & \text{otherwise}
    \end{cases}
\]
Moreover, the \defn{measure} of a term $\utm$ is written
$\meas{\utm}$ and defined as the pair $(\sizeVar{\utm},\size{\utm})$,
with the lexicographic order. 
\end{definition}

\begin{remark}
$\size{\utm} \geq \sizeVar{\utm}$
\end{remark}

For example,
$\sizeVar{(\lam{\var}{\utm})\,\var\,\vartwo\,\varthree} = \size{\lam{\var}{\utm}}$.

\begin{lemma}
\label{lemma:sizeVar_normalize}
\quad
\begin{enumerate}
\item
  If $\utm$ is in normal form
  then $\utm\,\var$ reduces to a normal form $\utmtwo$
  in at most one step,
  and moreover $\sizeVar{\utm} \geq \sizeVar{\utmtwo}$
  and $\size{\utm} \geq \size{\utmtwo}$.
\item
  If $\utm$ is in normal form
  then $\utm\,\var_1\hdots\var_n$ reduces to a normal form $\utmtwo$
  in at most $n$ steps,
  and moreover $\sizeVar{\utm} \geq \sizeVar{\utmtwo}$
  and $\size{\utm} \geq \size{\utmtwo}$.
\end{enumerate}
\end{lemma}
\begin{proof}
The second item is an easy consequence of the first one.
To show the first item,
we consider two cases, depending on whether $\utm$ has a redex at the root:
\begin{enumerate}
\item
  If $\utm\,\var$ has a redex at the root,
  then $\utm = \lam{\vartwo}{\utmtwo}$.
  There is a one-step reduction
  $\utm\,\var
  = (\lam{\vartwo}{\utmtwo)}\,\var
  \to \utmtwo\sub{\var}{\vartwo}$,
  where $\utmtwo\sub{\var}{\vartwo}$ is in normal form
  and
  $\sizeVar{\utm}
  = \sizeVar{\lam{\var}{\utmtwo}}
  = \size{\lam{\var}{\utmtwo}}
  > \size{\utmtwo}
  = \size{\utmtwo\sub{\var}{\vartwo}}
  \geq \sizeVar{\utmtwo\sub{\var}{\vartwo}}$.
\item
  If $\utm\,\var$ does not have a redex at the root,
  then $\utmtwo := \utm\,\var$ is already in normal form
  and $\sizeVar{\utm} = \sizeVar{\utm\,\var} = \sizeVar{\utmtwo}$.
\end{enumerate}
\end{proof}

\begin{lemma}
\label{lemma:sizeVar_neutral}
Let $\var\utm_1\hdots\utm_n$ be a term headed by a variable.
The for all $1 \leq i \leq n$
we have that $\meas{\var\utm_1\hdots\utm_n} > \meas{\utm_i}$.
\end{lemma}
\begin{proof}
By induction on $n$.
If $n = 0$, the property is vacuously true.
Assume the property holds for $n$ and let us check that the property holds
for $n + 1$. We consider two cases, depending on whether $\utm_{n+1}$ is
a variable:
\begin{enumerate}
\item
  If $\utm_{n+1}$ is a variable,
  then:
  \begin{enumerate}
  \item
    If $1 \leq i \leq n$, then:
    \[
      \begin{array}{rcll}
      &&
        \meas{\var\utm_1\hdots\utm_{n+1}}
      \\
      & = &
        (\sizeVar{\var\utm_1\hdots\utm_{n+1}}, \size{\var\utm_1\hdots\utm_{n+1}})
      \\
      & = &
        (\sizeVar{\var\utm_1\hdots\utm_{n}}, \size{\var\utm_1\hdots\utm_{n+1}})
        & \text{as $\utm_{n+1}$ is a variable}
      \\
      & > &
        (\sizeVar{\var\utm_1\hdots\utm_{n}}, \size{\var\utm_1\hdots\utm_{n}})
      \\
      & = &
        \meas{\var\utm_1\hdots\utm_{n}}
      \\
      & > &
        \meas{\utm_i}
        & \text{by \ih}
      \end{array}
    \]
  \item
    If $i = n + 1$, then:
    \[
      \begin{array}{rcll}
      &&
        \meas{\var\utm_1\hdots\utm_{n+1}}
      \\
      & = &
        (\sizeVar{\var\utm_1\hdots\utm_{n+1}}, \size{\var\utm_1\hdots\utm_{n+1}})
      \\
      & \geq &
        (1, \size{\var\utm_1\hdots\utm_{n+1}})
        & \text{as $\sizeVar{\var\utm_1\hdots\utm_{n+1}} \geq 1$}
      \\
      & > &
        (1, 1)
        & \text{as $n > 0$ so $\size{\var\utm_1\hdots\utm_{n+1}} > 1$}
      \\
      & = &
        (\sizeVar{\utm_{n+1}}, \size{\utm_{n+1}})
        & \text{as $\utm_{n+1}$ is a variable}
      \\
      & = &
        \meas{\utm_{n+1}}
      \end{array}
    \]
  \end{enumerate}
\item
  If $\utm_{n+1}$ is not a variable
  and $1 \leq i \leq n + 1$, then:
  \[
    \begin{array}{rcll}
    &&
      \meas{\var\utm_1\hdots\utm_{n+1}}
    \\
    & = &
      (\sizeVar{\var\utm_1\hdots\utm_{n+1}}, \size{\var\utm_1\hdots\utm_{n+1}})
    \\
    & = &
      (\size{\var\utm_1\hdots\utm_{n+1}}, \size{\var\utm_1\hdots\utm_{n+1}})
      & \text{as $\utm_{n+1}$ is not a variable}
    \\
    & > &
      (\size{\utm_i}, \size{\utm_i})
      & \text{as $\size{\var\utm_1\hdots\utm_{n+1}} > \size{\utm_i}$}
    \\
    & \geq &
      (\sizeVar{\utm_i}, \size{\utm_i})
    \\
    & = &
      \meas{\utm_i}
    \end{array}
  \]
\end{enumerate}
\end{proof}

\begin{theorem}[Completeness]
\label{thm:completeness}
If $\verif{\typ}{\subs{\utm}{\tenv}} \tos \iunit$
and $\utm$ is a pure term that reduces to a normal form,
then there exists a term $\tm$ such that $\judg{\tenv}{\tm}{\typ}$.
\end{theorem}
\begin{proof}
Let $\utm^\downarrow$ be the normal form of $\utm$.
We proceed by induction on $\meas{\utm^\downarrow}$.
Without loss of generality, let us write
$\typ = \typ_1 \to \hdots \typ_n \to \btyp$.
Note that
$\verif{\typ}{\subs{\utm}{\tenv}}
 \tos \verif{\btyp}{(\subs{\utm}{\tenv}\,\gen{\typ_1}\hdots\gen{\typ_n})}
 \tos \iunit$.
Let $\var_1,\hdots,\var_n$ be fresh variables,
and consider the term $\utm\,\var_1\hdots\var_n$
and the type environment $\tenvtwo := (\tenv,\var_1:\typ_1,\hdots,\var_n:\typ_n)$.
Note that $\subs{\utm}{\tenv} = \subs{\utm}{\tenvtwo}$ because $\var_1,\hdots,\var_n$
are fresh.
Then we have that
$\verif{\btyp}{\subs{(\utm\,\var_1\hdots\var_n)}{\tenvtwo}}
= \verif{\btyp}{(\subs{\utm}{\tenvtwo}\,\gen{\typ_1}\hdots\gen{\typ_n})}
= \verif{\btyp}{(\subs{\utm}{\tenv}\,\gen{\typ_1}\hdots\gen{\typ_n})}
\tos \iunit$.
Let $\utmtwo^\downarrow$ be the normal form of $\utm^\downarrow\,\var_1\hdots\var_n$.
Then by confluence (\cref{prop:confluence})
we have that
$\verif{\btyp}{\subs{(\utmtwo^\downarrow)}{\tenvtwo}} \tos \iunit$.
Note that $\utmtwo^\downarrow$ is pure and
that $\meas{\utm^\downarrow} \geq \meas{\utmtwo^\downarrow}$,
since $\sizeVar{\utm^\downarrow} \geq \sizeVar{\utmtwo^\downarrow}$
and $\size{\utm^\downarrow} \geq \size{\utmtwo^\downarrow}$
hold by \cref{lemma:sizeVar_normalize}.

Since $\utmtwo^\downarrow$ is a pure term in normal form, we consider two
cases, depending on whether it starts with an abstraction or it is of the form
$\var\,\utmtwo_1\hdots\utmtwo_m$:
\begin{enumerate}
\item
  If $\utmtwo^\downarrow$ starts with an abstraction,
  \ie $\utmtwo^\downarrow = \lam{\vartwo}{\utmthree}$,
  then we have that
  $\verifbtyp{\lam{\vartwo}{\subs{\utmthree}{\tenvtwo}}}
  = \verifbtyp{\subs{(\lam{\vartwo}{\utmthree})}{\tenvtwo}}
  = \verifbtyp{\subs{(\utmtwo^\downarrow)}{\tenvtwo}}
  \tos \iunit$,
  which is impossible,
  because the reducts of an abstraction are always abstractions,
  so in particular $\lam{\vartwo}{\subs{\utmthree}{\tenvtwo}}$
  cannot have $\genbtyp$ as a reduct.
\item
  If $\utmtwo^\downarrow = \var\,\utmtwo_1\hdots\utmtwo_m$
  then we have that
  $\verifbtyp{(\subs{\var}{\tenvtwo}\,\subs{\utmtwo_1}{\tenvtwo}\hdots\subs{\utmtwo_m}{\tenvtwo})}
  = \verifbtyp{\subs{(\var\,\utmtwo_1\hdots\utmtwo_m)}{\tenvtwo}}
  = \verifbtyp{\subs{(\utmtwo^\downarrow)}{\tenvtwo}}
  \tos \iunit$.
  We claim that $\var\in\dom{\tenvtwo}$.
  Indeed, if $\var\notin\dom{\tenvtwo}$, then $\subs{\var}{\tenvtwo} = \var$,
  and we have that
  $\verifbtyp{(\var\,\subs{\utmtwo_1}{\tenvtwo}\hdots\subs{\utmtwo_m}{\tenvtwo})}
   \tos \iunit$,
  which is impossible,
  because the reducts of a term headed by a variable $\var$
  are always headed by $\var$,
  so in particular $\var\,\subs{\utmtwo_1}{\tenvtwo}\hdots\subs{\utmtwo_m}{\tenvtwo}$
  cannot have $\genbtyp$ as a reduct.

  Then $\var \in \dom{\tenvtwo}$. Suppose then that $\var:\typtwo\in\tenvtwo$.
  Without loss of generality, $\typtwo$ is of the form
  $\typtwo = \typtwo_1\to\hdots\to\typtwo_k\to\btyptwo$.
  If $\var \notin \fv{\utm}$,
  let us abbreviate $\lamt{\typ}{\utm} := \lam{\var}{(\eunit{\verif{\typ}{\var}}{\utm})}$.
  Note that:
  \[
    \begin{array}{rlll}
      \verifbtyp{((\lamt{\typtwo_1}{\hdots\lamt{\typtwo_k}{\gen{\btyptwo}}})\,\subs{\utmtwo_1}{\tenvtwo}\hdots\subs{\utmtwo_m}{\tenvtwo})}
    & = &
      \verifbtyp{(\gen{\typtwo}\,\subs{\utmtwo_1}{\tenvtwo}\hdots\subs{\utmtwo_m}{\tenvtwo})}
    \\
    & = &
      \verifbtyp{(\subs{\var}{\tenvtwo}\,\subs{\utmtwo_1}{\tenvtwo}\hdots\subs{\utmtwo_m}{\tenvtwo})}
    \\
    & \tos &
      \iunit
    \end{array}
  \]
  We first claim that $k = m$.
  Let us discard the cases $k < m$ and $k > m$:
  \begin{enumerate}
  \item
    If $k < m$, then we obtain a contradiction, because we claim that
    $\verifbtyp{((\lamt{\typtwo_1}{\hdots\lamt{\typtwo_k}{\gen{\btyptwo}}})\,\subs{\utmtwo_1}{\tenvtwo}\hdots\subs{\utmtwo_m}{\tenvtwo})}$
    cannot reduce to $\iunit$.
    To show this, we consider two cases:
    \begin{enumerate}
    \item
      \label{thm:completeness:case_k_lt_m_stuck}
      If for every index $1 \leq i \leq m$ we have that
      $\verif{\typtwo_i}{\subs{\utmtwo_i}{\tenvtwo}} \tos \iunit$,
      then:
      \[
        \verifbtyp{((\lamt{\typtwo_1}{\hdots\lamt{\typtwo_k}{\gen{\btyptwo}}})\,\subs{\utmtwo_1}{\tenvtwo}\hdots\subs{\utmtwo_m}{\tenvtwo})}
        \tos
        \verifbtyp{(\gen{\btyptwo}\,\subs{\utmtwo_{k+1}}{\tenvtwo}\hdots\subs{\utmtwo_m}{\tenvtwo})}
      \]
      Note that the reducts of $\gen{\btyptwo}\,\subs{\utmtwo_{k+1}}{\tenvtwo}\hdots\subs{\utmtwo_m}{\tenvtwo}$
      always have $\gen{\btyptwo}$ at the head and exactly $m - k > 0$ arguments,
      so the argument of $\verifbtyp{\ARG}$ does not reduce to $\genbtyp$,
      and the term does not reduce to $\iunit$.
    \item
      Otherwise, let $1 \leq i \leq m$ be the least index such that
      $\verif{\typtwo_i}{\subs{\utmtwo_i}{\tenvtwo}}$ does not reduce to $\iunit$.
      Then:
      \[
        \begin{array}{ll}
        &
        \verifbtyp{((\lamt{\typtwo_1}{\hdots\lamt{\typtwo_k}{\gen{\btyptwo}}})\,\subs{\utmtwo_1}{\tenvtwo}\hdots\subs{\utmtwo_m}{\tenvtwo})}
        \\
        \tos &
        \verifbtyp{((\eunit{\verif{\typtwo_i}{\subs{\utmtwo_i}{\tenvtwo}}}{(\lamt{\typtwo_{i+1}}{\hdots\lamt{\typtwo_k}{\gen{\btyptwo}}})})\,\subs{\utmtwo_{i+1}}{\tenvtwo}\hdots\subs{\utmtwo_m}{\tenvtwo})}
        \end{array}
      \]
      Since $\verif{\typtwo_i}{\subs{\utmtwo_i}{\tenvtwo}}$ does not reduce to $\iunit$,
      the unit eliminator
      $(\eunit{\verif{\typtwo_i}{\subs{\utmtwo_i}{\tenvtwo}}}{\ARG})$
      does not reduce to a redex,
      so the argument of $\verifbtyp{\ARG}$ does not reduce to $\genbtyp$,
      and the term does not reduce to $\iunit$.
    \end{enumerate}
  \item
    If $k > m$, then we obtain a contradiction, because we claim that
    $\verifbtyp{((\lamt{\typtwo_1}{\hdots\lamt{\typtwo_k}{\gen{\btyptwo}}})\,\subs{\utmtwo_1}{\tenvtwo}\hdots\subs{\utmtwo_m}{\tenvtwo})}$
    cannot reduce to $\iunit$.
    To show this, note first that
    if there is an index $1 \leq i \leq k$ such that
    $\verif{\typtwo_i}{\subs{\utmtwo_i}{\tenvtwo}}$ does not reduce to $\iunit$,
    the situation is similar as for \cref{thm:completeness:case_k_lt_m_stuck} above.
    If for every index $1 \leq i \leq k$ we have that
    $\verif{\typtwo_i}{\subs{\utmtwo_i}{\tenvtwo}} \tos \iunit$,
    then:
    \[
      \verifbtyp{((\lamt{\typtwo_1}{\hdots\lamt{\typtwo_k}{\gen{\btyptwo}}})\,\subs{\utmtwo_1}{\tenvtwo}\hdots\subs{\utmtwo_m}{\tenvtwo})}
      \tos
      \verifbtyp{(\lamt{\typtwo_{m+1}}{\hdots\lamt{\typtwo_k}{\gen{\btyptwo}}})}
    \]
    Note that the reducts of $\lamt{\typtwo_{m+1}}{\ARG}$ always start with an abstraction,
    so the argument of $\verifbtyp{\ARG}$ does not reduce to $\genbtyp$,
    and the term does not reduce to $\iunit$.
  \end{enumerate}
  Hence $k = m$.
  If there is an index $1 \leq i \leq k$ such that
  $\verif{\typtwo_i}{\subs{\utmtwo_i}{\tenvtwo}}$ does not reduce to $\iunit$,
  the situation is similar as for \cref{thm:completeness:case_k_lt_m_stuck} above.
  Hence $\verif{\typtwo_i}{\subs{\utmtwo_i}{\tenvtwo}} \tos \iunit$
  for every index $1 \leq i \leq k$, and we have that:
  \[
    \verifbtyp{((\lamt{\typtwo_1}{\hdots\lamt{\typtwo_k}{\gen{\btyptwo}}})\,\subs{\utmtwo_1}{\tenvtwo}\hdots\subs{\utmtwo_m}{\tenvtwo})}
    \tos
    \verifbtyp{\gen{\btyptwo}}
  \]
  so $\verifbtyp{\gen{\btyptwo}} \tos \iunit$,
  which means that $\btyp = \btyptwo$.

  Recall that $\utmtwo^\downarrow$ is a pure term in normal form,
  and that $\utmtwo^\downarrow = \var\,\utmtwo_1\hdots\utmtwo_k$,
  so $\utmtwo_1,\hdots,\utmtwo_k$ are also pure terms in normal form.
  Note also that
  $\meas{\utm^\downarrow}
   \geq \meas{\utmtwo^\downarrow}
   > \meas{\utmtwo_i}$ for all $1 \leq i \leq k$,
  by resorting to \cref{lemma:sizeVar_neutral}.
  Since, for each $1 \leq i \leq k$,
  we know that $\verif{\typtwo_i}{\subs{\utmtwo_i}{\tenvtwo}} \tos \iunit$,
  by \ih we obtain that
  there exists a term $\tm_i$ such that
  $\judg{\tenvtwo}{\tm_i}{\typtwo_i}$.
  Since $\var:\typtwo_1\to\hdots\to\typtwo_k\to\btyp \in \tenv \subseteq \tenvtwo$,
  we can derive the judgment
  $\judg{\tenvtwo}{\var\,\tm_1\hdots\tm_k}{\btyp}$.
  Moreover, recall that $\tenvtwo = (\tenv,\var_1:\typ_1,\hdots,\var_n:\typ_n)$,
  so taking $\tm := \var\,\tm_1\hdots\tm_n$
  we have that
  $\judg{\tenv}{\lam{\var_1}{\hdots\lam{\var_n}{\var\,\tm_1\hdots\tm_k}}}{\typ_1\to\hdots\typ_n\to\btyp}$,
  as required.
\end{enumerate}
\end{proof}

